\makeatletter
\newcommand{\listintertext}{\@ifstar\listintertext@\listintertext@@}
\newcommand{\listintertext@}[1]{% \listintertext*{#1}
  \hspace*{-\@totalleftmargin}#1}
\newcommand{\listintertext@@}[1]{% \listintertext{#1}
  \hspace{-\leftmargin}#1}
\makeatother

\begin{center}
\section*{\centering SCHOOL OF ELECTRONICS, ELECTRICAL ENGINEERING and COMPUTER SCIENCE}

\subsection*{\centering CSC3002 – COMPUTER SCIENCE PROJECT}

\subsubsection*{\centering Dissertation Cover Sheet}

A signed and completed cover sheet must accompany the submission of the Software
Engineering dissertation submitted for assessment.


Work submitted without a cover sheet will NOT be marked.  

\begin{table}[h]
\centering
\begin{tabular}{|l|l|l|l|}
\hline
Student Name:  & Jamie Thompson    & Student Number:   & 40178456   \\ \hline
Project Title: & \multicolumn{3}{l|}{Kubernetes Custom Autoscaling} \\ \hline
Supervisor     & \multicolumn{3}{l|}{Dr. David Cutting}             \\ \hline
\end{tabular}
\end{table}

\subsubsection*{\centering Declaration of Academic Integrity}

\end{center}

Before submitting your dissertation please check that the submission:

\begin{enumerate}
    \item Has a full bibliography attached laid out according to the guidelines
    specified in the Student Project Handbook
    \item Contains full acknowledgement of all secondary sources used (paper-based and electronic)
    \item Does not exceed the specified page limit
    \item Is clearly presented and proof-read
    \item Is submitted on, or before, the specified or agreed due date. Late
    submissions will only be accepted in exceptional circumstances or where a
    deferment has been granted in advance.
\end{enumerate}

\textbf{
By submitting your dissertation you declare that you have completed the tutorial
on plagiarism at \url{http://www.qub.ac.uk/cite2write/introduction5.html} and
are aware that it is an academic offence to plagiarise.  You declare that the
submission is your own original work. No part of it has been submitted for any
other assignment and you have acknowledged all written and electronic sources
used.}

\begin{enumerate}
    \setcounter{enumi}{5}
    \item If selected as an exemplar, I agree to allow my dissertation to be
    used as a sample for future students. (Please delete this if you do not
    agree.)
\end{enumerate}

\begin{table}[h]
\centering
\begin{tabular}{|l|l|l|l|}
\hline
Student's Signature & Jamie Thompson & Date of Submission & 24/04/2020 \\ \hline
\end{tabular}
\end{table}

\newpage

\section*{Acknowledgements}

Making this project has taken a lot of time and work, I want to thank my
supervisor Dr. David Cutting for giving great guidance, and helping to push this
project in an interesting direction. His constant assistance and ability to
create opportunities for industry networking have been much appreciated.  

I am extremely appreciative of the support and encouragement from my family; my
parents Dawn and Ken, my brothers Justin and Scott and my sister Nadia, who all
helped push me along so I could finish this. 

\newpage

\section*{Abstract}

Autoscaling provides an automated process for managing demand for applications,
provisioning and distributing resources as required. Autoscaling allows seamless
elastic computing, with automated responses for scaling to meet demand quickly
and without user intervention.

Kubernetes autoscaling is limited by lack of customisation for the standard
autoscaler; the Horizontal Pod Autoscaler, and by the difficulty in creating a
customised autoscaler. A custom autoscaler requires an in-depth understanding of
the Kubernetes API and Kubernetes Pod Lifecycle - complex mechanics and
standards that are not directly relevant to autoscaling logic.

This project created a framework to address both of these issues by facilitating
a more simple approach to creating custom autoscalers - abstracting complexity
without ceding control or flexiblity. This simplification is with the ultimate
goal of allowing developers to create, deploy and share autoscalers with limited
Kubernetes API experience.

This project was successful in addressing these issues; meeting the requirements
laid out, allowing the development of both simple and complicated custom
autoscalers in any language, framework or environment, and has seen interest
from both the open source community and industry for use in development and
production environments.

\newpage
